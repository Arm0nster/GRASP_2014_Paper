\documentclass{IEEEtran}
\usepackage{graphicx}
\usepackage{amsmath}

\graphicspath{{./figures/}}

% \title{Integrating 3D Localization and Haptic Feedback for Interactive Graphics Manipulation}
% \title{Integrating 3D Localization, Haptic Devices, and Head Mounted Devices for Virtual Reality}
% \title{General Platform for Interactive Graphics Manipulation with Haptic Feedback via Large Scale Motion Capture}
\title{A General Platform for Immersive Virtual Reality and Interactive Manipulation with Haptic Feedback via Motion Capture}

\author{
    \IEEEauthorblockN{Armon Shariati} \\
\IEEEauthorblockA{Lehigh University \\ University of Pennsylvania}
}

\begin{document}
\setlength{\pdfpagewidth}{8.5in}
\setlength{\pdfpageheight}{11 in}
\maketitle

\begin{abstract}
    Most virtual reality systems generally fall into one of two categories:
    graphics oriented - which focus on developing immersive environments but
    generally fail to provide users with a strong sense of interactity - and
    haptics oriented - which focus on constructing tools for touch-based
    manipulation and auditory interaction but tend to be very specialized
    applications and tightly bound device ecosystems. We present our approach
    to creating immersive graphical settings with a high degree of interactive
    manipulation using any number of haptic devices.
\end{abstract}

\section{Introduction}
\label{sec:introduction}

Virtual reality has commanded the imagination of the media and the public for
decades, and in recent years the serious attention of researchers worldwide.
French playwright Antonin Artaud is first credited for coining the term
``virtual reality'' in his 1938 book \emph{The Theater and Its Double}. He uses
the term describe theater, whereby it is a reality that is both illusory and
purely fictitious \cite{website:popularizeVR}. The allure of VR, in the
technological sense, according to Michael Abrash, is that ``presence is an
incredibly powerful sensation, and it's unique to VR; there is no way to create
it in any other medium'' \cite{website:steampowered}, much like how Artuad felt
about theater. Many devices over the years try to impart the feeling of
presence but to do so is no trivial task, and for those that come close, the
sensation is truly magical. While VR has long been not much more than an
elusive science fiction, recent advancements in hardware and computing power
elevate its prospects to an attainable not-so-distant future. 

The prospects of creating these immersive amenable worlds which are a direct
extension of our own environments are seen to yield significant applications in
gaming, medicine, and defense, among others. Gaming is the most obvious example
of where VR might find application, and has consequently received the most
attention. Meanwhile, the later three exhibit a plethora of applications yet to
be addressed, and the solutions that are currently implemented have achieved
auspicious results. 

In a recent study at the Department of Surgery at Yale University, researchers
found that surgical residents receiving VR training for laparoscopic
cholecystectomy were able to perform gallbladder dissections 29\% faster than
the control group. Furthermore, non-VR-trained residents were nine times more
likely to transiently fail to make progress and five times more likely to
injure the gallbladder \cite{seymour2002virtual}.

The United States military has begun using fully immersive virtual simulation
training systems for soldiers in Fort Bragg, N.C. \cite{website:army}. They
found that ``the ability to train with this system allows the ``reset'' time to
be cut down, which allows the ability to get more repetitions in a shorter
amount of time and the ability to review each mission on a television screen to
enhance the after action review process upon completion of each mission''.  The
VR system lends itself to a mutable nature as well. ''The Semi-Automated Forces
Works station gives the trainer the option to create additional static items
like furniture and buildings or items that are animated such as dogs and birds,
inside the virtual world. There can also be modifications made during the
scenario like adding an improvised explosive device or more vehicles and
combatants.''

Head mounted devices (HMD), grew to become the most popular device for
delivering the VR experience for gamers.  These wearable devices track the
orientation of your head and use the measurements from their on board sensors
such as accelerometers and gyroscopes to adjust the view frame to correspond to
the real time movement of your head.  The first of such devices include
Nintendo's Virtual Boy, Virtual i-O i-glasses, and the VFX-1 Headgear from Forte
Tech. While all very groundbreaking for their time, they are all plagued with
the same problem as the countless HMD devices to follow, in that the processing
power could not contend with what the graphics demand called for. In addition
to low resolution displays and limited communication bandwidth, the VR
experience these devices provide are limited at best and would often cause
users to feel sick after a session \cite{zachara2009challenges}.

The most contemporary versions of these devices however, show a great deal of
promise. The Oculus Rift \cite{website:oculusvr} is by far the most promising
device in recent times. Founded by Palmer Luckey in 2012, Oculus has seen
meteoric success, raising \$2.5 million in it's kickstarter
campaign \cite{website:kickstarterovr}, and receiving praise and support from
gaming giants such as John Carmack of id Software and Michael Abrash of Valve.
But it's attention is not unwarranted since it is the first device of its kind
to carefully address the issue of tracker latency. The VR realism is directly
tied to this response time. To address this issue the original Development Kit
1 sampled the 3-axis gyros, accelerometers and magnetometers at rate of 250 Hz
\cite{website:roadtovr}.  The newest version of the Oculus Rift, the
Development Kit 2, features an internal tracking system with a sampling rate of
1000 Hz. Since the introduction of the Oculus to the market, many companies are
attempting to release their own competitive devices such as Sony's Project
Morpheus \cite{website:sony}, as well as ancillary input and haptic devices
such as the Virtuix Omni - an omnidirectional treadmill and Razer Hydra -
a haptic controller \cite{website:razer}.

As the visual aspects of VR are becoming ever closer to being solved, much more
needs to be done developing solutions for haptics, tracking, and input.  The
progress made thus far creating visually immersive environments are degraded if
users have no robust means of interaction. Much research has already been done
developing haptic devices to interact with VR systems, however many of these
solutions are large and stationary, require training to use, or simulate only
certain aspects of the experience. Many different types of devices exist today,
including exoskeletons, stationary devices, gloves and wearable devices,
specific task devices, and force feedback devices, to name a few.  Examples of
large commercial haptic systems to experience great success include the
CyberGrasp glove \cite{website:cybergrasp} and the phantom omni
\cite{website:geomagic}. Many game companies also attempt to embed haptic
feedback functionality into their game controllers as well such as the Nintendo
Wiimote \cite{website:wiimote}. 

Ultimately, there still exists a great divide between the graphic solutions and
haptic solutions as far as integration. While some systems attempt to close the
loop from perception, to interaction, to haptic feedback, they often make it
difficult to utilize advancements in the graphics technology and introduce
other tools into their ecosystem.  Moreover, many of these systems would have
great difficulty providing as large of a virtual space as required for some
military applications and provide as much precision for fine manipulations
tasks such as surgical training.

We present a solution for developing immersive and interactive VR environments,
which allows developers to efficiently leverage large scale motion caputre
systems in order to interface any number of haptic devices with a cutting edge
graphics pipeline, to achieve seamless interactivity between users and their
virtual realities. Our design also allows our system to interface with robotic
systems as well which could lead to interesting teleoperation and telepresence
applications.

We use a motion capture system to track a HMD in 3D volume as well as a haptic
device in order to manipulate the virtual environment. We have in place, a
basic demo which places the user in 5 meter long, 2.5 meter wide, and 2.5 meter
high squash court with a ball and paddle.  The pose of the paddle in virtual
space corresponds directly to the pose of the haptic device in the real world.
As the user hits the ball, he or she experiences the haptic feedback as the
ball experiences an impulse and travels along a proper trajectory.


\section{Head Mounted Display}
\label{sec:hmd}

\begin{figure}[!b]
\centering
\includegraphics[width=0.5\textwidth]{oculus.jpg}
\caption{Oculus Rift Development Kit 1. 
\cite{website:pcworld}}
\label{fig:oculus}
\end{figure}

\begin{figure}[]
\centering
\includegraphics[width=0.5\textwidth]{stereoscopic.jpg}
\caption{A stereoscopic split screen view also rendered with barrel distortion.
Frames displayed on the Oculus screen appear normal.}
\label{fig:stereo}
\end{figure}

Since the Oculus Rift seen in Fig. \ref{fig:oculus} is the current state of the
art in HMD VR technology, it is a natural choice as the cornerstone for
graphical immersion in our system. The most simple applications one can develop
for the Oculus are created using basic OpenGL. The only caveat being that a
separate framebuffer must be bound to the context for off-screen rendering. The
framebuffer must have a texture bound to at as well which will be written to at
the last stage of the conventional pipeline. Parameters for constructing the
texture are provided directly from the Oculus SDK programatically.  The Oculus
SDK will then take the data placed in the texture and perform the
post-processing for stereoscopic 3D rendering and distortion processing. The
Oculus Rift requires a scene to be rendered in split-screen stereo. The left
eye sees the left half of the screen and the right eye sees the right half.
Human eye pupils are approximately 65 mm apart.  This interpupillary distance
(IPD) must be taken into consideration for configuring the in-application
camera.  Therefore, each scene is rendered twice, once for the virtual camera
on the left, and once for the virtual camera on the right. Both of which are
subject to a translation with respect to one another causing the stereoscopic
effect. The lenses in the Oculus provide a large degree of magnification in
order to provide a wide field of view so as to enhance immersion. However, as a
result the image succumbs to a great deal of pincushion distortion. To rectify
this distortion, the software must apply an equal and opposite amount of barrel
distortion. The final image rendered to the Oculus's display can be seen in
Fig. \ref{fig:stereo}. For the remainder of this paper, when discussing the
camera view, this refers to the single camera model provided to the Oculus SDK
before the post processing occurs. 

Graphics is known as the inverse problem to computer vision. Where computer
vision seeks to extract information about an environment from an image,
graphics seeks to extract information to construct an image from an
environment. What gives computer graphics the illusion of position and depth is
a mathematical process of constructing a series of transformations taking
arbitrary points from one frame of reference to another to infer position and
ultimately a projective transform to infer depth. The location of all 
the particles in an object can be defined with respect to some inertial frame
of reference. If one observes the object from some alternative location and
orientation with respect to the same intertial frame, one can construct a 
transformation to take points from the inertial frame to the view frame. This
is already well understood by many who have experience with 3D graphics

% Fix the orientation of the axis to reflect graphical frame
% \begin{figure}[b!]
% \centering
% \includegraphics[width=0.5\textwidth]{world_view.png}
% \caption{Two frames of reference, the view and the world. The orienation and
% position of the view frame corresponds to the camera view.}
% \label{fig:worldview}
% \end{figure}


\section{Motion Capture}
\label{sec:mocap}

\begin{figure}[]
\centering
\includegraphics[width=0.5\textwidth]{view_mask.png}
\caption{On the left is a graphical representation of the Oculus with its
tracking rig attached as well as the two different coordinate axis
corresponding to the actual graphics view, and the mask. On the right
is a picture of the Oculus as it is used in the system.}
\label{fig:mask}
\end{figure}

Motion capture systems such as the Vicon system \cite{website:vicon}, which I
chose to use in the system,  track objects through a volume by calculating the
3D spatial displacements of a set of asymmetrically mounted markers. The
markers reflect the infrared radiation emitted by LEDs which in turn is the
only light that a set of cameras track after all other sources of light are
masked out. Using two or more cameras, the position of each marker can be found
via triangulation.

In order to best track the pose of the Oculus through the volume created by the
set of 6 cameras, one needs to mount the reflective markers in such a fashion
that they are difficult to occlude. To do so, I construct a simple lightweight
rig from balsa wood seen in Figure \ref{fig:mask}. I refer to the coordinate
frame associated with the rig as the mask frame. It is important to note that
the mask frame is not the same as the view frame. If orientation of the mask
frame was used for the orientation of the virtual camera, the displayed seen
would be incorrect. To account for this, I build another transformation to
take points from the mask frame to the view frame. I approximate the rotation
with the rotation matrix below.

\[
R = \begin{pmatrix} 0 & -1 & 0 \\ 0 & 0 & 1 \\ -1 & 0 & 0 \end{pmatrix}
\]

I approximate the translation between the two frames by using a ruler to
measure the distance from the point indicated as the center of the rig, and the
approximate center of the Oculus' front face.

Using a motion capture system to track position and rotation allows developers
to add as many extra devices to the system, provided there are a set of markers
mounted to the device which meet the necessary criterion. For my own demo I
chose to use a Nintendo Wiimote seen in Figure \ref{fig:wiimote}.

\begin{figure}[]
\centering
\includegraphics[width=0.5\textwidth]{wiimote.png}
\caption{Above is a Nintendo Wiimote with a set of four reflective markers
attached to its head such that it can be tracked through the motion capture
volume}
\label{fig:wiimote}
\end{figure}

\begin{figure}[]
\centering
\includegraphics[width=0.5\textwidth]{mocap.png}
\caption{This is a representation of what the different coordinate frames
look like in the system.}
\label{fig:mocap}
\end{figure}

A final representation of using motion capture to track the objects can be seen
in Figure \ref{fig:mocap}. One should note, that since the position and
orientation of the Oculus - and in turn the view frame - is now derived from
the Vicon, the graphics do not exist with respect to some arbitrary world frame
defined in the graphics world, but instead they are all defined with respect to
the origin defined by the Vicon system. The developer may choose where to set
the origin of the Vicon after they perform a calibration. Furthermore, the size
and position of all graphical objects are no longer an arbitrary unit length,
but are now millimeters, which is the unit reported by the Vicon. This grants
developers an extra degree of intuition because the virtual volume and all of
the virtual objects now directly correspond to space defined in the real world.


\section{Results}
\label{sec:results}

\begin{figure}[b!]
\centering
\includegraphics[width=0.5\textwidth]{reflection.png}
\caption{The trajectory of the ball incurs a reflection about the normal
vector defining the plane with which it makes contact.}
\label{fig:reflection}
\end{figure}

The test application I use to test the merit of the platform, includes three
major elements. It must provide an immersive graphical environment for the
user, interactivity with the environment, and finally some degree of haptic
feedback. The demo features a game of virtual squash. The user is placed within
a room which approximately corresponds size to the motion capture volume - 5
meters long, 2.5 meters wide, and 2.5 meters high. Alongside the user is a
paddle and ball. The pose of the paddle corresponds directly to the pose of the
Wiimote within the volume. See Figure \ref{fig:demo} for a snapshot.  When the
virtual paddle makes contact with the ball, the Wiimote vibrates, and the
paddle supplies the ball with an instantaneous velocity in the direction of the
normal vector defining the face of the paddle which made contact.  As the ball
travels, if it makes contact with any of the six faces of the bounding room or
any of the six faces of the paddle, the velocity of the ball is subject to a
reflection about the normal defining the face with which it made contact as
seen in Figure \ref{fig:reflection}. 

\[
\vec{v_{2}} = \vec{v_{1}} - 2 \frac{\vec{v_{i}} \bullet \vec{n}}{\vec{n} \bullet \vec{n}} \hat{n}
\]

Until the user chooses to reset the simulation, the ball will continue to 
bounce move about the room indefinitely.

\subsection{Software Architecture}

I built the application within the Robot Operating System (ROS) framework
\cite{website:ros}. ROS is a set of software libraries and tools that provide
``a structured communications layer above the host operating system of a
heterogeneous computing cluster'' \cite{quigley2009ros}. A ROS node publishes
the tracking data collected from the Vicon over a ROS topic which the main
graphics thread listens to. The application uses the pose of the Oculus and the
pose of the Wiimote provided by the Vicon, as the pose of the virutal camera
and virtual paddle respectively. The Vicon would provide measurements at a rate
of 100Hz. The graphics thread runs at an average rate of 60Hz, however, I
notice a sharp drop periodically down to 40Hz. The drop does not seem to affect
the rendering process greatly, but I would like to investigate the
cause\footnote{I suspect that the issue lies within communication with the
Wiimote. I naively placed the communication processes with the Wiimote within
the main graphics thread instead of making it its own process.}. The use of ROS
permits an ease of integration with many robotics platforms for teloperative
purposes, as many robotic systems are built within the ROS framework as well.

\subsection{Contact Modeling}

In order to simplify contact modeling I model the paddle as a rectangle even
though it is a rectangular prism. Therefore, two faces can really ever make
contact with the ball. The application determines contact with any face of the
room and the face of the paddle by continuously calculating the distance from
the ball to every face, and uses this set of measurements to ascertain whether
or not every point on the ball is within the bounding volume of the room and
whether or not any of the points on the ball exist on the face of the paddle.
Refer to Figure \ref{fig:normals} for an illustration of the different normals
associated with each face.

\[
D = \vec{(x - p)} \bullet \hat{n}
\]

Above, $D$ is the distance between a point and a plane; $x$ and $p$ are
the locations of a point on the plane and the point for which the calculation
is being performed, respectively.
\smallskip


\begin{figure}[]
\centering
\includegraphics[width=0.5\textwidth]{normals.png}
\caption{An illustration of the position and orientation of all the 
normals defining each face.}
\label{fig:normals}
\end{figure}

If every point on the ball is within the room, the distances between all the
points on the ball and each face of the room should be positive. If the sign of
any of the distances becomes negative, this indicates contact with the
associated face. The velocity vector of the ball then undergoes a reflection
about the normal vector affiliated with the face which makes contact.
Similarly, to determine contact with the paddle, the application determines if
any point on the ball has entered the bounding prism of the paddle. If so, the
application determines which side of our simplified face model the center of
the ball resides, and uses either the normal vector for reflection if the
center is in front of the paddle or the negative of the normal vector if the
center is behind the paddle.

\begin{figure}[]
\centering
\includegraphics[width=0.5\textwidth]{demo.png}
\caption{On the left is a single image of the environment during the simulation
which is rendered to the Oculus; notice the ball and paddle.  On the right is
the user interacting with the VR.}
\label{fig:demo}
\end{figure}

\section{Future Work}
\label{sec:future}

There are a few shortcomings of my system I would like to address as well 
as a few avenues for expansion in the future. 

The length of the wires connecting the Oculus to the desktop computer are
rather short, fettering the user with approximately 1.5 meter long tether.
Therefore, even though the generated volume has dimensions of 5 by 2.5 by 2.5
meters, the user only may move within an inscribed circle with a radius of 1.5
meters.  I would like to enable exploration of the entire volume in the future. 

Secondly, the environment the user is placed in - the virtual squash court - is
rather bland. There is no lighting or shadow. The ball, each wall of the room,
and each face of the paddle are all solid colors. Moreover, the main issue that
remains is that the environment is not very familiar to the user. I would
also like to spend more time developing more vivid and relatable scenes in order
to generate more immersive experiences. OpenGL may be sufficient, although
more sophisticated game engines such as Unity \cite{website:unity} may also
be of interest.

Finally, even though the platform is designed to support more sophisticated
haptic devices a Wiimote is the only haptic device used in the
system\footnote{Pressed for time, when there were interfacing issues with a
more advanced device, I defaulted to using a simple Wiimote.}. In the future I
would like to experiment with more advanced haptic devices and develop
simulations that would provide higher degrees of interactivity. Other
applications include, virtual sculpting, construction, and as mentioned before,
robotic teleoperation.


\section{Acknowledgements}
I would like to thank Rahul Mangharam for generously providing many of the
resources necessary for building the system. I would especially like to thank
my advisor Professor Camillo Jose Taylor and my graduate student mentor
Anthony Cowley for their tutelage. Finally, I would like extend special thanks
to Professor Katherine Kuchenbecker and Professor Max Mintz for their efforts
to organize the REU program.

\bibliographystyle{IEEEtran}
\bibliography{IEEEabrv,bibliography}

\end{document}
