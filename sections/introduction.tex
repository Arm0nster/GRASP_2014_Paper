\section{Introduction}
\label{sec:introduction}

Virtual reality has commanded the imagination of the media and the public for
decades, and in recent years the serious attention of researchers worldwide.
French playwright Antonin Artaud is first credited for coining the term
``virtual reality'' in his 1938 book \emph{The Theater and Its Double}. He uses
the term describe theater, whereby it is a reality that is both illusory and
purely fictitious \cite{website:popularizeVR}. The allure of VR, in the
technological sense, according to Michael Abrash, is that ``presence is an
incredibly powerful sensation, and it's unique to VR; there is no way to create
it in any other medium'' \cite{website:steampowered}, much like how Artuad felt
about theater. Many devices over the years try to impart the feeling of
presence but to do so is no trivial task, and for those that come close, the
sensation is truly magical. While VR has long been not much more than an
elusive science fiction, recent advancements in hardware and computing power
elevate its prospects to an attainable not-so-distant future. 

The prospects of creating these immersive amenable worlds which are a direct
extension of our own environments are seen to yield significant applications in
gaming, medicine, and defense, among others. Gaming is the most obvious example
of where VR might find application, and has consequently received the most
attention. Meanwhile, the later three exhibit a plethora of applications yet to
be addressed, and the solutions that are currently implemented have achieved
auspicious results. 

In a recent study at the Department of Surgery at Yale University, researchers
found that surgical residents receiving VR training for laparoscopic
cholecystectomy were able to perform gallbladder dissections \%29 faster than
the control group. Furthermore, non-VR-trained residents were nine times more
likely to transiently fail to make progress and five times more likely to
injure the gallbladder \cite{seymour2002virtual}.

The United States military has begun using fully immersive virtual simulation
training systems for soldiers in Fort Bragg, N.C. \cite{website:army}. They
found that ``the ability to train with this system allows the ``reset'' time to
be cut down, which allows the ability to get more repetitions in a shorter
amount of time and the ability to review each mission on a television screen to
enhance the after action review process upon completion of each mission''.  The
VR system lends itself to a mutable nature as well. ''The Semi-Automated Forces
Works station gives the trainer the option to create additional static items
like furniture and buildings or items that are animated such as dogs and birds,
inside the virtual world. There can also be modifications made during the
scenario like adding an improvised explosive device or more vehicles and
combatants.''

Head mounted devices (HMD), grew to become the most popular device for
delivering the VR experience for gamers.  These wearable devices track the
orientation of your head and use the measurements from their on board sensors
such as accelerometers and gyroscopes to adjust the view frame to correspond to
the real time movement of your head.  The first of such devices include
Nintendo's Virtual Boy, Virtual i-O i-glasses, and the VFX-1 Headgear from Forte
Tech. While all very groundbreaking for their time, they are all plagued with
the same problem as the countless HMD devices to follow, in that the processing
power could not contend with what the graphics demand called for. In addition
to low resolution displays and limited communication bandwidth, the VR
experience these devices provide are limited at best and would often cause
users to feel sick after a session \cite{zachara2009challenges}.

The most contemporary versions of these devices however, show a great deal of
promise. The Oculus Rift \cite{website:oculusvr} is by far the most promising
device in recent times. Founded by Palmer Luckey in 2012, Oculus has seen
meteoric success, raising \$2.5 million in it's kickstarter
campaign \cite{website:kickstarterovr}, and receiving praise and support from
gaming giants such as John Carmack of id Software and Michael Abrash of Valve.
But it's attention is not unwarranted since it is the first device of its kind
to carefully address the issue of tracker latency. The VR realism is directly
tied to this response time. To address this issue the original Development Kit
1 sampled the 3-axis gyros, accelerometers and magnetometers at rate of 250 Hz
\cite{website:roadtovr}.  The newest version of the Oculus Rift, the
Development Kit 2, features an internal tracking system with a sampling rate of
1000 Hz. Since the introduction of the Oculus to the market, many companies are
attempting to release their own competitive devices such as Sony's Project
Morpheus \cite{website:sony}, as well as ancillary input and haptic devices
such as the Virtuix Omni - an omnidirectional treadmill and Razer Hydra -
a haptic controller \cite{website:razer}.

As the visual aspects of VR are becoming ever closer to being solved, much more
needs to be done developing solutions for haptics, tracking, and input.  The
progress made thus far creating visually immersive environments are degraded if
users have no robust means of interaction. Much research has already been done
developing haptic devices to interact with VR systems, however many of these
solutions are large and stationary, require training to use, or simulate only
certain aspects of the experience. Many different types of devices exist today,
including exoskeletons, stationary devices, gloves and wearable devices,
specific task devices, and force feedback devices, to name a few.  Examples of
large commercial haptic systems to experience great success include the
CyberGrasp glove \cite{website:cybergrasp} and the phantom omni
\cite{website:geomagic}. Many game companies also attempt to embed haptic
feedback functionality into their game controllers as well such as the Nintendo
Wiimote \cite{website:wiimote}. 

Ultimately, there still exists a great divide between the graphic solutions and
haptic solutions as far as integration. While some systems attempt to close the
loop from perception, to interaction, to haptic feedback, they often make it
difficult to utilize advancements in the graphics technology and introduce
other tools into their ecosystem.  Moreover, many of these systems would have
great difficulty providing as large of a virtual space as required for some
military applications and provide as much precision for fine manipulations
tasks such as surgical training.

We present a solution for developing immersive and interactive VR environments,
which allows developers to efficiently leverage large scale 3D localization in
order to interface any number of haptic devices with a cutting edge graphics
pipeline, to achieve seamless interactivity between users and their virtual
realities. Our design also allows our system to interface with robotic systems
as well which could lead to interesting teleoperation and telepresence
applications.

We use a motion capture system to track a HMD in 3D volume as well as a haptic
device in order to manipulate the virtual environment. We have in place, a
basic demo which places the user in 5 meter long, 2.5 meter wide, and 2.5 meter
high squash court with a ball and paddle.  The pose of the paddle in virtual
space corresponds directly to the pose of the haptic device in the real world.
As the user hits the ball, he or she experiences the haptic feedback as the
ball experiences an impulse and travels along a proper trajectory.

