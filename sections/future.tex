\section{Future Work}
\label{sec:future}

There are a few shortcomings of my system I would like to address as well 
as a few avenues for expansion in the future. 

The length of the wires connecting the Oculus to the desktop computer are
rather short, fettering the user with an approximately 1.5 meter long tether.
Therefore, even though the virtual space has dimensions of 5 by 2.5 by 2.5
meters, the user only may move within an inscribed circle with a radius of 1.5
meters.  I would like to enable exploration of the entire volume in the future. 

Secondly, the environment - the virtual squash court - is rather bland. There
is no lighting or shadow. The ball, each wall of the room, and each face of the
paddle are all solid colors. Moreover, the main issue that remains is that the
environment is not very familiar to the user. I would like to spend more time
developing more vivid and relatable scenes in order to generate more immersive
experiences. OpenGL may be sufficient, although more sophisticated game engines
such as Unity \cite{website:unity} may also be of interest.

Finally, even though the platform is designed to support more sophisticated
haptic devices a Wiimote is the only haptic device used in the
system\footnote{Pressed for time, when there were interfacing issues with a
more advanced device, I defaulted to using a simple Wiimote.}. In the future I
would like to experiment with more advanced haptic devices and develop
simulations that would provide higher degrees of interactivity. Other
simulations of interest include, virtual sculpting, construction, and as
mentioned before, robotic teleoperation.
