\section{Results}
\label{sec:results}

\begin{figure}[b!]
\centering
\includegraphics[width=0.5\textwidth]{reflection.png}
\caption{The trajectory of the ball incurs a reflection about the normal
vector defining the plane with which it makes contact.}
\label{fig:reflection}
\end{figure}

The test application I use to test the merit of the platform, includes three
major elements. It must provide an immersive graphical environment for the
user, interactivity with the environment, and finally some degree of haptic
feedback. The demo features a game of virtual squash. The user is placed within
a room which approximately corresponds size to the motion capture volume - 5
meters long, 2.5 meters wide, and 2.5 meters high. Alongside the user is a
paddle and ball. The pose of the paddle corresponds directly to the pose of the
Wiimote within the volume. See Figure \ref{fig:demo} for a snapshot.  When the
virtual paddle makes contact with the ball, the Wiimote vibrates, and the
paddle supplies the ball with an instantaneous velocity in the direction of the
normal vector defining the face of the paddle which made contact.  As the ball
travels, if it makes contact with any of the six faces of the bounding room or
any of the six faces of the paddle, the velocity of the ball is subject to a
reflection about the normal defining the face with which it made contact as
seen in Figure \ref{fig:reflection}. 

\[
\vec{v_{2}} = \vec{v_{1}} - 2 \frac{\vec{v_{i}} \bullet \vec{n}}{\vec{n} \bullet \vec{n}} \hat{n}
\]

Until the user chooses to reset the simulation, the ball will continue to 
bounce move about the room indefinitely.

\subsection{Software Architecture}

I built the application within the Robot Operating System (ROS) framework
\cite{website:ros}. ROS is a set of software libraries and tools that provide
``a structured communications layer above the host operating system of a
heterogeneous computing cluster'' \cite{quigley2009ros}. A ROS node publishes
the tracking data collected from the Vicon over a ROS topic which the main
graphics thread listens to. The application uses the pose of the Oculus and the
pose of the Wiimote provided by the Vicon, as the pose of the virutal camera
and virtual paddle respectively. The Vicon would provide measurements at a rate
of 100Hz. The graphics thread runs at an average rate of 60Hz, however, I
notice a sharp drop periodically down to 40Hz. The drop does not seem to affect
the rendering process greatly, but I would like to investigate the
cause\footnote{I suspect that the issue lies within communication with the
Wiimote. I naively placed the communication processes with the Wiimote within
the main graphics thread instead of making it its own process.}. The use of ROS
permits an ease of integration with many robotics platforms for teloperative
purposes, as many robotic systems are built within the ROS framework as well.

\subsection{Contact Modeling}

In order to simplify contact modeling I model the paddle as a rectangle even
though it is a rectangular prism. Therefore, two faces can really ever make
contact with the ball. The application determines contact with any face of the
room and the face of the paddle by continuously calculating the distance from
the ball to every face, and uses this set of measurements to ascertain whether
or not every point on the ball is within the bounding volume of the room and
whether or not any of the points on the ball exist on the face of the paddle.
Refer to Figure \ref{fig:normals} for an illustration of the different normals
associated with each face.

\[
D = \vec{(x - p)} \bullet \hat{n}
\]

Above, $D$ is the distance between a point and a plane; $x$ and $p$ are
the locations of a point on the plane and the point for which the calculation
is being performed, respectively.
\smallskip


\begin{figure}[]
\centering
\includegraphics[width=0.5\textwidth]{normals.png}
\caption{An illustration of the position and orientation of all the 
normals defining each face.}
\label{fig:normals}
\end{figure}

If every point on the ball is within the room, the distances between all the
points on the ball and each face of the room should be positive. If the sign of
any of the distances becomes negative, this indicates contact with the
associated face. The velocity vector of the ball then undergoes a reflection
about the normal vector affiliated with the face which makes contact.
Similarly, to determine contact with the paddle, the application determines if
any point on the ball has entered the bounding prism of the paddle. If so, the
application determines which side of our simplified face model the center of
the ball resides, and uses either the normal vector for reflection if the
center is in front of the paddle or the negative of the normal vector if the
center is behind the paddle.

\begin{figure}[]
\centering
\includegraphics[width=0.5\textwidth]{demo.png}
\caption{On the left is a single image of the environment during the simulation
which is rendered to the Oculus; notice the ball and paddle.  On the right is
the user interacting with the VR.}
\label{fig:demo}
\end{figure}
