\section{Motion Capture}
\label{sec:mocap}

\begin{figure}[]
\centering
\includegraphics[width=0.5\textwidth]{view_mask.png}
\caption{On the left is a graphical representation of the Oculus with its
tracking rig attached as well as the two different coordinate axis
corresponding to the actual graphics view, and the mask. On the right
is a picture of the Oculus as it is used in the system.}
\label{fig:mask}
\end{figure}

Motion capture systems such as the Vicon system \cite{website:vicon}, which I
chose to use in my system,  track objects through a volume by calculating 3D
spatial displacements of a set of asymmetrically mounted markers. The markers
reflect the infrared radiation emitted by LEDs which in turn is the only light
that is tracked by a set of cameras which mask out all other sources of light.
Using two or more cameras, the position of each marker can be found via
triangulation.

In order to best track the pose of the Oculus through the volume created by the
set of 6 cameras, one needs to mount the reflective markers in such a fashion
that they are difficult to occlude. To do so, I construct a simple lightweight
rig from balsa wood seen in Figure \ref{fig:mask}. I refer to the coordinate
frame associated with the rig as the mask frame. It is important to note that
the mask frame is not the same as the view frame. If orientation of the mask
frame was used for the orientation of the virtual camera, the displayed seen
would be incorrect. To account for this, I build another transformation to
take points from the mask frame to the view frame. I approximate the rotation
with the rotation matrix below.

\[
R = \begin{pmatrix} 0 & -1 & 0 \\ 0 & 0 & 1 \\ -1 & 0 & 0 \end{pmatrix}
\]

I approximate the translation between the two frames by using a ruler to
measure the distance from the point indicated as the center of the rig, and the
approximate center of the Oculus' front face.

Using a motion capture system to track position and rotation allows us to
add as many extra devices to the system, provided there are a set of markers
mounted to the device which meet the necessary criterion. For my own demo 
I chose to use a Nintendo Wiimote seen in Figure \ref{fig:wiimote}.

\begin{figure}[]
\centering
\includegraphics[width=0.5\textwidth]{wiimote.png}
\caption{Above is a Nintendo Wiimote with a set of four reflective markers
attached to its head such that it can be tracked through the motion capture
volume}
\label{fig:wiimote}
\end{figure}

\begin{figure}[]
\centering
\includegraphics[width=0.5\textwidth]{mocap.png}
\caption{This is a representation of what the different coordinate frames
look like in the system.}
\label{fig:mocap}
\end{figure}

A final representation of using motion capture to track the objects can be seen
in Figure \ref{fig:mocap}. One should note, that since now the position and
orientation of the Oculus - and in turn the view frame - is derived from the
Vicon, which establishes a world coordinate frame at a point defined on the
ground, the graphics do not exist with respect to some arbitrary world frame
defined in the graphics world, but instead they are all defined with respect to
the origin defined by the Vicon system. Furthermore, the size and position of
all graphical objects are no longer an arbitrary unit length, but are now
millimeters which is the unit reported by the Vicon. This grants developers an
extra degree of intuition because the virtual volume and all of the virtual
objects now directly correspond to space defined in the real world.


